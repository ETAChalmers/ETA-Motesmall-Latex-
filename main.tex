% ---------- % Mall för ETAs föreningsmöten % ---------- %
% Detta är en mall till ETAs föreningsmötesprotokoll. Fyll i settings i vilket möte det är samt vilka som är ansvariga. Där finns även ett gäng trevliga kortkommandon.
% Sista ändring av: Johan "Hank" Karlsson 2019-02-07
% Författare: Johan "Hank" Karlsson och styrelsen 15/16


% ---------- % Start på dokument % ---------- %
% LaTeX-Settings %
% Here the packages and settings are changed.  
% Last edit by Johan "Hank" Karlsson 2019-02-07

% ---------- % Packages % ---------- %
\documentclass[a4paper, 12pt]{article}
\usepackage[utf8]{inputenc}
\usepackage{moreverb}						% List settings
\usepackage[swedish]{babel}                          
\usepackage[pdftex]{graphicx}               % Required to import graphic files
\usepackage{graphicx}						% Figures
\usepackage{subfig}							% Enables subfigures
\usepackage{fancyhdr}
\usepackage{t1enc} 
\usepackage{pdfpages}
\usepackage{listings}
\usepackage{float} 		                    % Enables object position enforcement using [H]
\usepackage{parskip}						% Enables vertical spaces correctly 
\usepackage{eso-pic}						% Create cover page background

% ---------- % Settings % ---------- %
\pagestyle{fancy}
\topmargin -20.0pt
\headheight 56.0pt
\setcounter{section}{1}
\newcounter{bil}                            % För bilagor
\setcounter{bil}{1}


% ---------- % TAG BORT DENNA NÄR SIDNUMRERING BEHÖVS % ---------- %
\pagenumbering{gobble}
% ---------- % TAG BORT DENNA NÄR SIDNUMRERING BEHÖVS % ---------- %


% ---------- % Custom commands to be changed each time % ---------- %
\newcommand{\verksamhetsar}{<Verksamhetsår>}    % <Verksamhetsår>
\newcommand{\datummote}{ <Mötets datum>}         % <Mötets datum>
\newcommand{\datumnu}{<Dagens datum>}           % <Dagens datum>
\newcommand{\nr}{<Vilket möte>}                 % <Vilket möte>
\newcommand{\justone}{<Justerare 1>}           % <Justerare #1>
\newcommand{\justtwo}{<Justerare 2>}           % <justerare #2>
\newcommand{\ordf}{<Ordförande>}                % <Ordförande>
\newcommand{\sekr}{<Sekreterare>}               % <Sekreterare>

% ---------- % Custom commands % ---------- %
\newcommand{\beslut}{\flushright \textbf{[Beslut]} \flushleft}

\newcommand{\bilaga}{\textbf{[Bilaga \thebil]}\stepcounter{bil}}

\newcommand{\justin}[2][]{\flushleft \textbf{#2 #1 justerades in.}}

\newcommand{\justut}[2][]{\flushleft \textbf{#2 #1 justerades ut.}}

\newcommand{\sect}[1][]{\section*{\S \thesection. #1}
    \addcontentsline{toc}{section}{\S \thesection  \hspace{4pt}  #1}    % För dagordnings kommando
    \stepcounter{section} 
} 

\newcommand{\ssect}[1][]{\subsection*{#1}
    % \addcontentsline{toc}{subsection}{ \thesubssection  \hspace{4pt}  #1}    % För dagordnings kommando
    \stepcounter{subsection} 

}

\newcommand{\para}{\paragraph \noindent}


\newcommand{\kortsig}[1]{
  \begin{flushright}
    \parbox{150pt}{\flushright\hrulefill\\#1}\hspace*{2\bigskipamount}
  \end{flushright}}
  
\newcommand{\fullsig}{
  \parbox{200pt}{\hrulefill\\Datum\hfill Underskrift \hfill   Namnförtydligande}}
  
\newcommand{\dagordning}{
    \renewcommand*\contentsname{Dagordning}
    \tableofcontents

}

  % ---------- % Header % ---------- %
\renewcommand{\headrule}{\vbox to 0pt{\hfill\hbox to 400pt {\hrulefill}}}

\lhead{
    \raisebox{0pt}[-1000pt][0pt]{\includegraphics[width=90pt]{figure/eta.jpg} }
    \parbox[b]{200pt}{ E-sektionens Teletekniska Avdelning\\
    Chalmers studentkår} 
}

\rhead{ 
    \flushright Sidan \thepage\ av \pageref{LastPage}\\
    \datumnu 
}

% ---------- % Footer % ---------- %
\renewcommand{\footrulewidth}{\headrulewidth}

\lfoot{\flushleft
    \begin{tabular}{l}\\
    \makebox[1.2in]{\hrulefill} \\ % adds space between the two sets of signatures
         \sekr
    \end{tabular}
    }
    
\rfoot{ \flushright 
\begin{tabular}{l}\\
    \makebox[1.2in]{\hrulefill} \\% adds space between the two sets of signatures
         \justtwo
    \end{tabular}
}

\cfoot{\noindent\begin{tabular}{ll}\\
    \makebox[1.2in]{\hrulefill} & \makebox[1.2in]{\hrulefill} \\[0.5ex]% adds space between the two sets of signatures
         \ordf & \justone \\[1ex]
    \end{tabular}
}


\begin{document}
% ---------- % Rubrik % ---------- %
\section*{\center Protokoll <möte> \datummote}
Kl: 17:17\\
Mötesnummer: \nr \ \verksamhetsar\\
Plats: EL43\\
Kallelse: se bifogad.\\ \\
Närvarande:\\ \\

\begin{tabular}{l l l}
\itshape Namn & \itshape Namn & \itshape Namn\\
  & & \\

\end{tabular}\\

\sect[Mötets raska öppnande]

Mötet öppnats klockan XX:XX av <ordf>.


\sect[Val av justeringsmän tillika rösträknare]

XXX och XXX nominerade sig själva frivilligt.

Mötet röstar enhälligt

\emph{\textbf{att} välja XXX och XXX som justeringsmän tillika rösträknare.}




\sect[Mötets stadgeenliga utlysande/mötets beslutsmässighet]

Mötet utlystes på ETA:s Facebook genom ett event och kallelsen har suttit på styrelsendörren (skrubben), med mycket god marginal; mer än två veckor innan mötestillfället. Samt närvarar tillräckligt många medlemmar för att vara ett stadgeenligt möte.

Mötet beslutar enhälligt

\emph{\textbf{att} mötet är stadgeenligt utlyst och beslutsmässigt.}





\sect[Närvarorätt och yttranderätt för gästande]

Vid mötets öppnande närvarade inga gästande, dock var det ett antal väntade gäster.

Mötet beslutar enhälligt

\emph{\textbf{att} om  gästande skulle tillkomma skulle de få närvarorätt dock ej äga rösträtt på dagens möte.}




\sect[Val av mötesordförande och mötessekreterare]


Sittande, XXX och XXX, nominerades till \newline mötesordförande respektive mötessekreterare. 

Mötet beslutar enhälligt

\emph{\textbf{att} välja XXX till mötesordförande,}\\
\emph{\textbf{att} välja XXX till mötessekreterare.}



\newpage

\sect[Fastställande av dagordning]

XXX föreslog att under \emph{§X motioner} placera \emph{Motion } som den första motionen.

Mötet beslutar enhälligt

\emph{\textbf{att} under ''§X  motioner'' placera ''Motion'' som den första motionen,}

\emph{\textbf{att} godkänna dagordningen med rådande ändringar.}


\sect[Föregående mötesprotokoll och uppföljning av beslut]




\sect[Verksamhetsberättelser]

Styrelsen från 2010/2011 gav sin verksamhetsberättselse genom XXX, och XXX avgav med entusiasm, taktfast och munter stämma verksamhetsberättelsen från år 2015/2016. 


Mötet beslutar enhälligt

\emph{\textbf{att} godkänna verksamhetsberättelsen från ETA:s styrelse år 2010/2011,}

\emph{\textbf{att} godkänna verksamhetsberättelsen från ETA:s styrelse år 2015/2016.}


\sect[Föredrag av balans-och resultaträkning.]

XXX gick kort igenom hur bra auktionen gått och påpekade att vi med goda marginaler har pengar till alla propositioner och motioner som tillkommit.


\sect[Revisionsberättelse]

Samtliga revisionsberättelser bordlägges då de ej är färdigställda. Bordlagda beslut tas upp under nästkommande möte. Styrelsen från 2012/2013 och 2015/2016 jobbar på deras revisionsberättelser då dessa är ej färdigställda.


Mötet beslutar enhälligt

\emph{\textbf{att} bordlägga revisionsberättelsen från ETA:s styrelse år 2010/2011,}

\emph{\textbf{att} bordlägga revisionsberättelsen från ETA:s styrelse år 2012/2013,}

\emph{\textbf{att} bordlägga revisionsberättelsen från ETA:s styrelse år 2015/2016.}


\sect[Ansvarsfriheter]

Från redovisade verksamhetsrapporter och revisionsberättelser åligger det mötet att besluta huruvida föregående års styrelser kan tilldelas ansvarsfrihet. Bordlagda beslut tas åter upp på nästkommande möte.

Mötet beslutar enhälligt

\emph{\textbf{att} bordlägga beslutet gällande ansvarsfrihet för ETA:s styrelse år 2010/2011,}

\emph{\textbf{att} bordlägga beslutet gällande ansvarsfrihet för ETA:s styrelse år 2012/2013,}

\emph{\textbf{att} bordlägga beslutet gällande ansvarsfrihet för ETA:s styrelse år 2015/2016.}

\sect[Aktuell ekonomisk rapport]

\ssect[Auktionsresultat]

\ssect[Aktuellt resultat]

\ssect[Budget för kvällens motioner]



\sect[Val av valkommitté för styrelsen 2019/2020]

Nuvarande styre bestående XXX nomineras till att agera valkommitté för styrelsen 2017/2018.

 Mötet beslutar enhälligt
 
 \emph{\textbf{att} tillsätta XXX som valkommitté för ETA:s styrelse 2017/2018.}





\sect[Propositioner]
Till dagens möte har en proposition samt fyra motioner inkommit. Samtliga \linebreak motioner finns bifogade till mötesprotokollet.


\begin{itemize}

    \item Propositionen \emph{Proposition } berör ...

\end{itemize}

<beskrivning>

Mötet beslutade enhälligt

\emph{\textbf{att} godkänna propositionen  ''Proposition ETAs rullande soffa'' med förändringen att det skulle vara en totalbudget på 13 000 SEK, exklusive sponsring.}



\sect[Motioner]


\begin{itemize}

    \item Motionen \emph{Motion } berör ...

\end{itemize}






\newpage
\textbf{Motioner}

    \emph{Motion,}
 
    Mötet beslutade enhälligt
    
    \emph{\textbf{att} godkänna motionen ''Motion '' i sin helhet.}
    
    \vspace{0.2in}
    
    \emph{Motion, }
    
    Mötet beslutade enhälligt
    
    \emph{\textbf{att} godkänna motionen ''Motion'' i sin helhet.}

    \vspace{0.2in}

  
\sect[Övriga frågor]

   
\sect[Mötets avslutande]
Mötet avslutade klockan 20:09 av Vanessa Vannas.


% ---------- % Dagordning % ---------- %
\newpage
\cleardoublepage
\pagebreak
\dagordning

% ---------- % Underskrifter % ---------- %
\newpage
Underskrifter


\noindent\begin{tabular}{l}\\[8ex]
\makebox[4in]{\hrulefill} \\[1ex]% adds space between the two sets of signatures
Sekreterare \sekr\\[8ex]

\makebox[4in]{\hrulefill} \\[1ex]% adds space between the two sets of signatures
Mötesordförande \ordf\\[8ex]     

\makebox[4in]{\hrulefill} \\[1ex]% adds space between the two sets of signatures
Justeringsman \justone\\[8ex]

\makebox[4in]{\hrulefill} \\[1ex]% adds space between the two sets of signatures
Justeringsman \justtwo\\    

\end{tabular}

\label{LastPage}





\newpage
\cleardoublepage
\pagebreak



% ---------- % Appendix % ---------- %
\appendix
% \includepdf[pages=-]{Kallelse_ETA.pdf}



\end{document}