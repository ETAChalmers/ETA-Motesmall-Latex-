% ---------- % Mall för ETAs föreningsmöten % ---------- %
% Detta är en mall till ETAs föreningsmötesprotokoll. Fyll i settings i vilket möte det är samt vilka som är ansvariga. Där finns även ett gäng trevliga kortkommandon.
% Sista ändring av: Johan "Hank" Karlsson 2019-02-07
% Författare: Johan "Hank" Karlsson och styrelsen 15/16


% ---------- % Start på dokument % ---------- %
% LaTeX-Settings %
% Here the packages and settings are changed.  
% Last edit by Johan "Hank" Karlsson 2019-02-07

% ---------- % Packages % ---------- %
\documentclass[a4paper, 12pt]{article}
\usepackage[utf8]{inputenc}
\usepackage{moreverb}						% List settings
\usepackage[swedish]{babel}                          
\usepackage[pdftex]{graphicx}               % Required to import graphic files
\usepackage{graphicx}						% Figures
\usepackage{subfig}							% Enables subfigures
\usepackage{fancyhdr}
\usepackage{t1enc} 
\usepackage{pdfpages}
\usepackage{listings}
\usepackage{float} 		                    % Enables object position enforcement using [H]
\usepackage{parskip}						% Enables vertical spaces correctly 
\usepackage{eso-pic}						% Create cover page background

% ---------- % Settings % ---------- %
\pagestyle{fancy}
\topmargin -20.0pt
\headheight 56.0pt
\setcounter{section}{1}
\newcounter{bil}                            % För bilagor
\setcounter{bil}{1}


% ---------- % TAG BORT DENNA NÄR SIDNUMRERING BEHÖVS % ---------- %
\pagenumbering{gobble}
% ---------- % TAG BORT DENNA NÄR SIDNUMRERING BEHÖVS % ---------- %


% ---------- % Custom commands to be changed each time % ---------- %
\newcommand{\verksamhetsar}{<Verksamhetsår>}    % <Verksamhetsår>
\newcommand{\datummote}{ <Mötets datum>}         % <Mötets datum>
\newcommand{\datumnu}{<Dagens datum>}           % <Dagens datum>
\newcommand{\nr}{<Vilket möte>}                 % <Vilket möte>
\newcommand{\justone}{<Justerare 1>}           % <Justerare #1>
\newcommand{\justtwo}{<Justerare 2>}           % <justerare #2>
\newcommand{\ordf}{<Ordförande>}                % <Ordförande>
\newcommand{\sekr}{<Sekreterare>}               % <Sekreterare>

% ---------- % Custom commands % ---------- %
\newcommand{\beslut}{\flushright \textbf{[Beslut]} \flushleft}

\newcommand{\bilaga}{\textbf{[Bilaga \thebil]}\stepcounter{bil}}

\newcommand{\justin}[2][]{\flushleft \textbf{#2 #1 justerades in.}}

\newcommand{\justut}[2][]{\flushleft \textbf{#2 #1 justerades ut.}}

\newcommand{\sect}[1][]{\section*{\S \thesection. #1}
    \addcontentsline{toc}{section}{\S \thesection  \hspace{4pt}  #1}    % För dagordnings kommando
    \stepcounter{section} 
} 

\newcommand{\ssect}[1][]{\subsection*{#1}
    % \addcontentsline{toc}{subsection}{ \thesubssection  \hspace{4pt}  #1}    % För dagordnings kommando
    \stepcounter{subsection} 

}

\newcommand{\para}{\paragraph \noindent}


\newcommand{\kortsig}[1]{
  \begin{flushright}
    \parbox{150pt}{\flushright\hrulefill\\#1}\hspace*{2\bigskipamount}
  \end{flushright}}
  
\newcommand{\fullsig}{
  \parbox{200pt}{\hrulefill\\Datum\hfill Underskrift \hfill   Namnförtydligande}}
  
\newcommand{\dagordning}{
    \renewcommand*\contentsname{Dagordning}
    \tableofcontents

}

  % ---------- % Header % ---------- %
\renewcommand{\headrule}{\vbox to 0pt{\hfill\hbox to 400pt {\hrulefill}}}

\lhead{
    \raisebox{0pt}[-1000pt][0pt]{\includegraphics[width=90pt]{figure/eta.jpg} }
    \parbox[b]{200pt}{ E-sektionens Teletekniska Avdelning\\
    Chalmers studentkår} 
}

\rhead{ 
    \flushright Sidan \thepage\ av \pageref{LastPage}\\
    \datumnu 
}

% ---------- % Footer % ---------- %
\renewcommand{\footrulewidth}{\headrulewidth}

\lfoot{\flushleft
    \begin{tabular}{l}\\
    \makebox[1.2in]{\hrulefill} \\ % adds space between the two sets of signatures
         \sekr
    \end{tabular}
    }
    
\rfoot{ \flushright 
\begin{tabular}{l}\\
    \makebox[1.2in]{\hrulefill} \\% adds space between the two sets of signatures
         \justtwo
    \end{tabular}
}

\cfoot{\noindent\begin{tabular}{ll}\\
    \makebox[1.2in]{\hrulefill} & \makebox[1.2in]{\hrulefill} \\[0.5ex]% adds space between the two sets of signatures
         \ordf & \justone \\[1ex]
    \end{tabular}
}


\begin{document}

% ---------- % Rubrik % ---------- %
\section*{\center Protokoll <möte> \datummote}
Kl: 17:17\\
Mötesnummer: \nr \ \verksamhetsar\\
Plats: EL43\\
Kallelse: se bifogad.\\ \\
Närvarande:\\ \\

\begin{tabular}{l l l}
\itshape Namn & \itshape Namn & \itshape Namn\\
  & & \\

\end{tabular}\\

\sect[Mötets raska öppnande]

Mötet öppnats klockan XX:XX av <ordf>.


\sect[Val av justeringsmän tillika rösträknare]

XXX och XXX nominerade sig själva frivilligt.

Mötet röstar enhälligt

\emph{\textbf{att} välja XXX och XXX som justeringsmän tillika rösträknare.}




\sect[Mötets stadgeenliga utlysande/mötets beslutsmässighet]

Mötet utlystes på ETA:s Facebook genom ett event och kallelsen har suttit på styrelsendörren (skrubben), med mycket god marginal; mer än två veckor innan mötestillfället. Samt närvarar tillräckligt många medlemmar för att vara ett stadgeenligt möte.

Mötet beslutar enhälligt

\emph{\textbf{att} mötet är stadgeenligt utlyst och beslutsmässigt.}





\sect[Närvarorätt och yttranderätt för gästande]

Vid mötets öppnande närvarade inga gästande, dock var det ett antal väntade gäster.

Mötet beslutar enhälligt

\emph{\textbf{att} om  gästande skulle tillkomma skulle de få närvarorätt dock ej äga rösträtt på dagens möte.}




\sect[Val av mötesordförande och mötessekreterare]


Sittande, XXX och XXX, nominerades till \newline mötesordförande respektive mötessekreterare. 

Mötet beslutar enhälligt

\emph{\textbf{att} välja XXX till mötesordförande,}\\
\emph{\textbf{att} välja XXX till mötessekreterare.}



\newpage

\sect[Fastställande av dagordning]

Felix Mannerhagen föreslog att under \emph{§13 Propositioner och motioner} placera \emph{Motion pallstaplare} som den första motionen.

Mötet beslutar enhälligt

\emph{\textbf{att} under ''§13 Propositioner och motioner'' placera ''Motion pallstaplare'' som den första motionen,}

\emph{\textbf{att} godkänna dagordningen med rådande ändringar.}


\sect[Föregående mötesprotokoll och uppföljning av beslut]


Från föregående mötesprotokoll finns ingenting bordlagt för mötet att behandla. Sedan föregående möte har det införskaffats en ny svarv av modellen "örn" årsmodell 1970, av tillverkaren storebro.



\sect[Ekonomisk redovisning och uppdatering]

Marcus Arvidsson gick kort igenom hur bra auktionen gått och påpekade att vi med goda marginaler har pengar till alla propositioner och motioner som tillkommit.






\sect[Verksamhetsberättelser]

Styrelsen från 2010/2011 gav sin verksamhetsberättselse genom XXX, och XXX avgav med entusiasm, taktfast och munter stämma verksamhetsberättelsen från år 2015/2016. 


Mötet beslutar enhälligt

\emph{\textbf{att} godkänna verksamhetsberättelsen från ETA:s styrelse år 2010/2011,}

\emph{\textbf{att} godkänna verksamhetsberättelsen från ETA:s styrelse år 2015/2016.}


\sect[Revisionsberättelser]
Samtliga revisionsberättelser bordlägges då de ej är färdigställda. Bordlagda beslut tas upp under nästkommande möte. Styrelsen från 2012/2013 och 2015/2016 jobbar på deras revisionsberättelser då dessa är ej färdigställda.


Mötet beslutar enhälligt

\emph{\textbf{att} bordlägga revisionsberättelsen från ETA:s styrelse år 2010/2011,}

\emph{\textbf{att} bordlägga revisionsberättelsen från ETA:s styrelse år 2012/2013,}

\emph{\textbf{att} bordlägga revisionsberättelsen från ETA:s styrelse år 2015/2016.}



\sect[Ansvarsfriheter]
Från redovisade verksamhetsrapporter och revisionsberättelser åligger det mötet att besluta huruvida föregående års styrelser kan tilldelas ansvarsfrihet. Bordlagda beslut tas åter upp på nästkommande möte.

Mötet beslutar enhälligt

\emph{\textbf{att} bordlägga beslutet gällande ansvarsfrihet för ETA:s styrelse år 2010/2011,}

\emph{\textbf{att} bordlägga beslutet gällande ansvarsfrihet för ETA:s styrelse år 2012/2013,}

\emph{\textbf{att} bordlägga beslutet gällande ansvarsfrihet för ETA:s styrelse år 2015/2016.}





\sect[Val av valkommitté för styrelsen 2017/2018]

Nuvarande styre bestående XXX nomineras till att agera valkommitté för styrelsen 2017/2018.

 Mötet beslutar enhälligt
 
 \emph{\textbf{att} tillsätta XXX som valkommitté för ETA:s styrelse 2017/2018.}





\sect[Propositioner och motioner]
Till dagens möte har en proposition samt fyra motioner inkommit. Samtliga \linebreak motioner finns bifogade till mötesprotokollet.

\begin{itemize}

    \item Propositionen \emph{Proposition ETAs rullande soffa} berör inköp av komponenter till en eventuell ny rullande soffa.

    \item Motionen \emph{Motion pallstaplare} berör införskaffandet av en handdriven pallstaplare då vår eldrivna har gått i pension.

    \item Motionen \emph{Motion, Svarvverktyg} berör införskaffandet av ett antal verktyg till svarvarna.

    \item Motionen \emph{Motion, Rundmatningsbord} berör införskaffandet av ett nytt rundmatningsbord.

    \item Motionen \emph{Motion digitaler till pelarborren} berör införskaffandet av digital-mätutrustning till pelarborren.
\end{itemize}


\textbf{Propositioner}

    \emph{ETAs rullande soffa}\\
Efter en diskussion konstaterades det att vissa siffror i motionerandet saknades, därmed yrkade Sebastian Sahlin på att ändra på motionen till att innefatta en budget på 13 000 SEK till skillnad från dess originalbelopp. Yrkandet lades då det motionerades separat för varje del till soffan och det framkom att vissa kostnader även saknades.

Alexander Davidsson påpekade att vi skulle ångra att vi köpte komponenter från Kina då de kunde vara dåliga kopior. 



Mötet beslutade enhälligt

\emph{\textbf{att} godkänna propositionen  ''Proposition ETAs rullande soffa'' med förändringen att det skulle vara en totalbudget på 13 000 SEK, exklusive sponsring.}


\newpage
\textbf{Motioner}

    \emph{Motion pallstaplare}
    
    Felix Mannerhagen inledde med att förklara att en truck med dött batteri som läcker olja är av föga nytta och att det som resultat kommer behövas investeras i en ny pallyftanordning.
    
    Kort diskussion råder om huruvida det kan vara tungt med en handdriven truck eller för lågt i tak, varpå Jacob Rosén påpekade att vi inte bör halvrumpa det.
    
    Mötet beslutade enhälligt
    
    \emph{\textbf{att} godkänna motionen ''Motion pallstaplare'' i sin helhet.}
    
    \vspace{0.2in}
    
    \emph{Motion, Svarvverktyg}
    
    Felix Mannerhagen gav en kort introduktion till sin motion där det framkom frågor om vad ''skär'' är för något och vad det kan användas till vilket enkelt besvarades.
    

    Mötet beslutade enhälligt
    
    \emph{\textbf{att} godkänna motionen ''Motion, Svarvverktyg'' i sin helhet.}

    \vspace{0.2in}

    \emph{Motion, Rundmatningsbord}  
    
    Felix Mannerhagen gav en kort genomgång av rundmatningsbord och dess nytta varpå en kort diskussion följde.
    

    Mötet beslutade enhälligt
    
    \emph{\textbf{att} godkänna motionen ''Motion, Rundmatningsbord'' i sin helhet.}
    
   \vspace{0.2in}
    
    \emph{Motion digitaler till pelarborren}
    
    Felix Mannerhagen gick kort igenom nyttan och möjligheterna följt av en Kort diskussion. 
         
    Mötet beslutade enhälligt
    
    \emph{\textbf{att} godkänna motionen ''Motion digitaler till pelarborren'' i sin helhet.}




\sect[Övriga frågor]

Fråga uppkom på hur mycket senaste auktionen drog in. Årets kassör, Markus Arvidsson, meddelar att från årets auktion har intäkter inkommit på ungefär 117 000:- och att det var en mycket lyckad auktion.

Fråga uppkom på hur vi gör med den för nuvarande trasiga 3D-printern. Marcus Arvidsson har kvittorna och Jacob Andersson tar kontakt med Max Sikström angående garantiärende och service, då hans bror jobbar på företaget där 3D-printern köptes ifrån.

Materialförvaltaren Mikael Bengtsson låter meddela att det kommer att vara ett hackathon tillsammans med Mikrofabriken med flera. Mer information kommer vid ett senare tillfälle.

Förseningar har uppkommit i radiocertifikatsprovet, men preliminärt kommer provet hållas i mars. Mer information kommer vid ett senare tillfälle.

Sebastian Sahlin låter meddela att vädret börjar bli bättre och att med vårvärmen kommer det att sättas upp en ny yagi-antenn. Denna kan användas till roliga saker på 2m-bandet, vilket innebär att man kan prata mer med folk i Sverige.

Mötet informeras om att diskussion pågår med SK6AW om en framtida radiolänk till Brudaremossen på något GHz-band.

Marcus Arvidsson påpekade att ETA fått ett QSL-kort från godtycklig plats. Diskussion om vart kortet kom ifrån uppkom men det framkom inte vart ifrån, Ryssland och Frankrike var i fokus. 

Mötet informeras av Frans-Erik Isaksson att arbetet på soffan i Spårvagnen går framåt, dock påpekas det att assistans skulle främja soffans förmjuknande.
    

   
\sect[Mötets avslutande]
Mötet avslutade klockan 20:09 av Vanessa Vannas.

% ---------- % Underskrifter % ---------- %
\newpage
Underskrifter


\noindent\begin{tabular}{l}\\[8ex]
\makebox[4in]{\hrulefill} \\[1ex]% adds space between the two sets of signatures
Sekreterare \sekr\\[8ex]

\makebox[4in]{\hrulefill} \\[1ex]% adds space between the two sets of signatures
Mötesordförande \ordf\\[8ex]     

\makebox[4in]{\hrulefill} \\[1ex]% adds space between the two sets of signatures
Justeringsman \justone\\[8ex]

\makebox[4in]{\hrulefill} \\[1ex]% adds space between the two sets of signatures
Justeringsman \justtwo\\    

\end{tabular}

\label{LastPage}

\newpage
\cleardoublepage
\pagebreak


% ---------- % Appendix % ---------- %
\appendix
% \includepdf[pages=-]{Kallelse_ETA.pdf}



\end{document}